%audio Lettering parte2 mercoledi 30 ottobre, 23 min
Analisi lettera x lettera (x tavola, 1cm x modulo e 3 mm per tratto lettera)
- L
- E: braccio centrale APPOGGIATO appena sopra al centro geometrico; braccio centrale piu piccolo di quello sup
- F: braccio centrale CENTRATO con il centro geometrico
- K: occhio centro geometrico != ottico (le aste oblique nn sono alla metà, ma appena sopra) e asta sup nn arriva alla fine
- B: cerchio inferiore > superiore; occhiello sotto esce a dx
- S: parte sup e inferiore escono appena fuori dalla griglia
- P e R: occhiello esce leggermente a dx; per la R la gamba termina fuori
- J: parte bassa esce 
- V: vertice basso esce leggermente
- A: vertice alto esce leggermente e barra è un po sotto la metà
- H: barra APPOGGIA sulla linea di metà
- T
- U:parte bassa esce leggermente
- N: punta a sx e dx vertici escono leggermente
- X: occhio centro ottico parte alta le diagonali NON arrivano ai bordi
- Y: nulla di importante, incontro tra obliqui è al centro
- Z: occhio parte alta e centro
- O: sborda da tutti i lati
-  Q: uguale a O
- C: come O ma vedi tavola
- G: uguale a C
- D: 
- M: vedi tavola (la linea magenta vale anche sulla parte dx)
- W: vertici alti e bassi sbordano
